
\begin{enumerate}[label=\alph*)]

    \item Viele Hinweise erleichtern das Abrufen der Information im Gedächtnis.
Das Erinnern enthält weniger Hinweise als das Erkennen und ist deshalb
eher schwieriger. Beim Erkennen, merkt man, dass ein Ereignis
oder eine Information vertraut ist und beim Erinnern muss man Details
aus dem Gedächtnis abrufen. 
    
    \item Generell macht es das Bedienen von Programmen einfacher für den Menschen, wenn er 
    gewisse Schemata \textit{erkennen} kann. Wenn der Benutzer ein Schema oder ein Muster der Bedienung
    erkennt, benötigt er nicht so viel Speicher. Was sicher auch nützlich ist, ist es sich an universellen 
    Dingen zu orientieren, beispielsweise Icons welche schon existieren, sodass der Benutzer sie leicht erkennt.
    
    \item Chunks werden innerhalb von 15-30 Sekunden wieder vergessen, deshalb ist diese Aussage 
    sinnlos. Das Menü wäre an sich ohnehin nicht sehr sinnvoll wenn man sich die Menü-Items merken müsste als User.
    Es würde also nur dann Sinn ergeben wenn man das Menü innerhalb 30 Sekunden immer wieder benutzen müsste,
    was aber auch eher mühsam scheint. Ausserdem kann man im KZG sowieso mehr als 4 Chunks speichern.
    Anders gesehen macht es eher Sinn ein Menü auf 4 oder 5 Items zu beschränken, dass es übersichtlich ist und 
    der Mensch diese sofort bewusst wahrnehmen kann. 
    
\end{enumerate}
