
\begin{enumerate}[label=\alph*)]

    \item Das sensorische Gedächtnis wird auch Ultrakurzzeitgedächtnis genannt, da es Sinneseindrücke wie
das Hören oder Sehen zwar aufnimmt aber quasi sofort wieder verwirft, da diese als nicht
wichtig empfunden werden. Was auf dem sensorischen Gedächtnis gespeichert wird, wird 
ständig von neuen Inputs überschrieben. \\
    Das Kurzzeitgedächtnis (KZG) hingegen kann Inhalte etwas länger speichern (ca. 15 Sekunden) und kann auch
länger bestehen bleiben wenn man den Inhalt immer wieder erneuert. Der Mensch muss dafür
seine volle Aufmerksamkeit auf das Erneuern des Inhaltes richten, denn das KZG ist sehr empfindlich
gegenüber Störungen.  \\
Zusammengefasst ist das KZG für den temporären Abruf von Informationen da, 
während beim sensorischen Gedächtnis eine gewisse Persistenz nach dem empfangenen Reiz bleibt, 
die Information jedoch nicht gespeichert wird. 

    \item Das Langzeitgedächtnis (LZG) beschreibt eigentlich unser ganzes 'Wissen', also Fakten, Verhaltensregeln,
Erfahrungswissen, etc. Es hat also im Gegensatz zum KZG eine viel grössere
Kapazität. Man unterscheidet im LZG das episodische (Erinnerungen in serieller Form) und
semantische (Fakten, Konzepte die man sich angeeignet hat) Gedächtnis. Die Zugriffszeit auf das LZG
ist langsamer als beim KZG und etwas aus dem LZG zu vergessen ist unwahrscheinlicher und erfolgt
auch langsamer.
    
\end{enumerate}

 
 

