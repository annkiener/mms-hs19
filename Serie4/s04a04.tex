Ein \textit{Irrtum} kommt auf, wenn Menschen sich auf Erinnerungen 
bezüglich vergangener Ereignisse anstatt systematischer Analyse 
verlassen. Wenn also die Situation falsch klassifiziert und nicht 
alle relevanten Faktoren in Betracht gezogen werden. Sie passieren 
bei Anfängern und zeigen meist, dass dem User das mentale Modell 
des Systems fehlt. Ein Irrtum tritt beim Vergleich der Resultate 
mit den Erwartungen auf, also auf den höheren Ebenen der Kognition, 
während ein \textit{Ausrutscher} in der Wahrnehmung oder 
Interpretation eines Ergebnis passiert, also auf den niederen 
Ebenen der Kognition. Bei einem Ausrutscher ist die Handlung, 
die ausgeführt wird, nicht die, die eigentlich beabsichtigt war. 
Es gibt 2 Arten von Ausrutschern: Handlungsbasierte Ausrutscher, 
bei der die falsche bzw. richtige Handlung auf dem richtigen bzw. 
falschen Objekt ausgeführt wird, oder Gedächtnisverluste, bei 
welchen durch Vergessen die Handlung nicht komplett durchgeführt 
werden kann.


Ein Designer kann das Auftreten von Irrtümern minimieren, 
indem er Bedienelemente eindeutig kennzeichnet, sodass die 
Situation kaum anders klassifiziert werden kann, als das, was 
sie eben ist. Das Design sollte gute Affordance, Signifiers, 
Mappings, Constrains und Feedbacks haben.


Das Auftreten von Ausrutschern kann dagegen minimiert werden 
indem Buttons oder Icons mit genügend Abstand platziert werden 
und es eindeutig und klar ist, was sie bedeuten. Die Sequenzen 
der Dialoge sollten unterscheidbar sein und das ‘done it’-Gefühl 
muss dem User zum richtigen Zeitpunkt gegeben werden.
