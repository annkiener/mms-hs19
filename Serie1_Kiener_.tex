\documentclass{article}

% encoding of the document
\usepackage[utf8]{inputenc}

% encoding of font, umlauts should work now
\usepackage[T1]{fontenc}

% main language
\usepackage[ngerman]{babel}

\usepackage{bookmark}


% page size
% \usepackage{showframe}
\usepackage{geometry}
\geometry{a4paper}

% no identation after new paragraph
\usepackage{parskip}

% better tables
\usepackage{tabularx}

% quotes
\usepackage[autostyle=true,german=quotes]{csquotes}

% links
\usepackage{hyperref}

\usepackage{graphicx}

\usepackage{tikzsymbols}

% custom title
\usepackage{titlesec}
\titleformat{\section}%
[hang]% <shape>
{\normalfont\bfseries\Large}% <format>
{}% <label>
{0pt}% <sep>
{}% <before code>

\begin{document}
    % title
    \title{Mensch Maschine Schnittstelle \\ HS2019 \\ Übungsserie 1}
    \date{\today}
    \author{Ann Kiener 17-115-916 \\ Noe}
    \maketitle

    \section{Aufgabe 1}
    \begin{itemize}
        \item Beispiel 1: Ann ist der Benutzer eines Game Boy. Ihre Aufgabe ist es 
        den Highscore bei Pacman zu knacken. Ihr Umfeld ist ein Zug, denn sie spielt
        am liebsten während dem Pendeln. Das Werkzeug ist der Game Boy und die
        Schnittstelle die Pfeiltasten und das A-, bzw. B-Tastenfeld.
        \item Beispiel 2: Noe ist der Benutzer eines Klaviers. Seine Aufgabe ist es
        \textit{Für Elise} möglichst korrekt und schön zu spielen. Sein Umfeld ist
        das Musikzimmer in seiner Wohnung. Das Werkzeug ist das Klavier und die 
        Schnittstelle ist die Klaviatur.
    \end{itemize}

    \section{Aufgabe 2}
   Bush entwirft das Konzept der Wissensmaschine Memex (Memory Extender), 
    die als Vorläufer eines PC's und einer Hypertext-Maschine gilt. Memex ist ein Gerät,
    auf dem ein Mensch all seine Daten speichern kann, seine Bücher, handschriftliche Aufzeichnungen, Abbildungen etc.
     Dadurch wird die Maschine zu einem „enlarged intimate supplement“ des menschlichen Gedächtnisses. Man könnte meinen,
     er beschreibt das heutige Smartphone.\\
     Ähnlichkeiten zum Smartphone hat die Memex insofern, dass sie auch individuell ist,
     also ein Gerät ist auf eine Person abgestimmt. Man kann fast alles darauf abspeichern,
     wie beim Smartphone die Musik, Bücher, Kommunikation etc. 
     Bush beschreibt die Memex auch als verfügbar aus einer beliebigen Distanz, also ähnlich
     zum Smartphone, welches immer und überall verfügbar ist. Dies scheint schwierig vorzustellen,
     da nach Bush's äusserlichen Beschreibung der Memex (Form eines Schreibtisches), es kaum möglich ist,
     dies überall verfügbar zu machen. 
     Erwähnt wird aber auch die Möglichkeit von Tonaufnahmen, bzw. die simultane Umwandlung von Sprache in Schrift und insbesondere
     verfügt die Memex über eine Indexierung des Speichers also eine Durchsuchbarkeitsfunktion. Vergleichbar
     ist dies mit einem Browserverlauf.
    

    \section{Aufgabe 3}


    \section{Aufgabe 4}
    \begin{itemize}
        \item Beispiel 1: Die Erfindung der Maus, welche es ermöglicht mit Handbewegungen
        eine grafische Oberfläche zu bedienen. So versetzte sich der Fokus der Forschung in HCI
        sicher mehr in Richtung dieser grafischer Benutzerschnittstelle.

        \item Beispiel 2: Andersherum erlaubt die Einführung eines Touchscreens eine
        noch direktere Steuerung einer Benutzeroberfläche. Diese Erfindung macht somit eine 
        Maus nicht mehr zwingend nötig. Damit wurden auch Begriffe wie 'Multi-Touch' wichtig -
        Gerät erkennt gewisse Gesten wie wischen zum weiterblättern, zoomen durch 'Ziehen und Fallenlassen, etc. 
        
    \end{itemize}

    \section{Aufgabe 5}

\end{document}

