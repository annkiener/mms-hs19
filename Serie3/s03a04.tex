\begin{enumerate}[label=\alph*)]

    \item  \begin{itemize}
                 \item Das \textit{Konzeptmodell} stellt auf vereinfachte Weise dar, wie etwas funktioniert. 
                 Dies muss weder vollständig noch präzise sein, solange es nützlich ist. Es kann aber auch detaillierter 
                 und komplex sein wenn man beispielsweise solche in technischen Betriebsanleitungen betrachtet.

                 \item Das \textit{mentale Modell} ist ein Konzeptmodell welches sich im Verstand des Menschen befindet.
                 Ohne ein gutes mentales Modell bedienen Menschen die Applikationen blindlings und auswendig. Ein gutes 
                 mentales Modell ist also wichtig für die Bedienung eine richtige Interaktion mit einem Produkt. Es kann
                 Mensch zu Mensch verschieden sein.

                 \item Das \textit{System-Image} ist sozusagen die Grundlage zur Bildung eines guten mentalen Modells. Es 
                 beschreibt die Zusammensetzung aus verschiedenen Informationen, wie wahrnehmbare Hinweise (Affordance, Signifiers,
                 Constraints, Mappings und auch Feedback) darauf, wie etwas funktioniert. Unter anderem gehört zum System-Image auch
                das Benutzer sich schon auskennen mit gewissen Produkten durch Benutzen von ähnlichen Geräten/Technologien und dass es 
                die wichtigen Informationen auch in Werbungen und Katalogen, Betriebsanleitung oder Ähnliches gibt. Die Benutzer vertrauen auf
                all diese Informationen zur Bildung eines guten mentalen Modells und zur richtigen Benutzung eines Produktes.
  \end{itemize}
  
        Zusammengefasst: Das mentale Modell im Verstand des Menschen ist eine Art Konzeptmodell und bildet sich aus dem System-Image
        
  

    \item  Das System-Image ist sozusagen der Übermittler des Konzeptmodells eines Designer zum mentalen Modell eines Benutzers.
    Ziel ist es, dass Konzeptmodell und mentales Modell übereinstimmen. Da Designer und Benutzer aber nicht miteinander 
    kommunizieren können muss das System-Image zusammenhängend, angemessen und vollständig sein,
    sodass der Benutzer ein gutes mentales Modell bilden kann. Das System-Image spielt also eine zentrale Rolle für die
    richtige Interaktion des Nutzers mit dem Produkt.


\end{enumerate}