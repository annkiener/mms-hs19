Der Benutzer möchte eine Datei auf seinem Schreibtisch umbennen. Wir nehmen an,
dass dieser Benutzer ein macOS System verwendet. Wir verwenden
das Interaktionsmodell von Norman um diesen Ablauf zu analysieren:

\begin{itemize}
  \item Ziel: Umbennen der Datei Untitled.txt zu Titled.txt
  \item Planen: Die Datei mit dem Namen Untitled.txt werde ich umbennen zu
    Titled.txt.
  \item Spezifizieren: 
    \begin{enumerate}
      \item Terminal öffnen
      \item Zum Schreibtisch wechseln: 'cd Desktop'
      \item Verfizieren, dass Datei vorhanden ist: 'ls -lah | grep
        Unititled.txt'
      \item Datei umbennen: 'mv Untitled.txt Titled.txt'
      \item Verfizieren, dass Datei umbennent wurde: 'ls -lah | grep Titled.txt'
    \end{enumerate}
  \item Ausführen:
    \begin{enumerate}
      \item Terminal öffnen
      \item Zum Schreibtisch wechseln: 'cd Desktop'
      \item Verfizieren, dass Datei vorhanden ist: 'ls -lah | grep
        Unititled.txt'
      \item Datei umbennen: 'mv Untitled.txt Titled.txt'
      \item Verfizieren, dass Datei umbennent wurde: 'ls -lah | grep Titled.txt'
    \end{enumerate}
  \item Wahrnehmen: Die Datei Untitled.txt ist auf dem Schreibtisch nicht mehr
    vorhanden und neu gibt es eine Datei Titled.txt
  \item Interpretieren: Der "Name" der Datei wurde geändert.
  \item Vergleich: Befindet sich eine Datei namens Titled.txt auf dem
    Schreibtisch? 
\end{itemize}

Bei dieser Umsetztung ist die \textit{Gulf of execution} für einen unerfahrenen
Benutzer sehr hoch, da Textinterfaces zu Beginn sehr wenig \textit{Affordance}
und \textit{Signifiers} liefern. Hingegen hat die Interaktion via Terminal
relativ grosse \textit{Contrains} und führt zu einem guten Konzeptmodell.

Die \textit{Gulf of evalutaion} ist relativ gering, mit wenig Mühe z.B. dem
Command 'ls -lah | grep Titled.txt' oder noch einfacher dem Blick auf das
Schreibtisch GUI, kann der Benutzer verfizieren, dass die Datei tatsächlich
umbennent wurde. 
    

