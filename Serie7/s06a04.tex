\begin{itemize}

    \item Struktur und Protokollführung
    
    Der Experte beginnt mit einer Begrüssung, erläutert wichtige Informationen
    zum Ablauf des Tests und erklärt der Testperson was sie machen muss (ehrlich sein,
    alles laut aussprechen, etc.)
    \\
    Danach wird eine Einverständniserklärung unterschrieben, dass es für die
    Testperson in Ordnung ist, dass sie aufgenommen wird.
    \\
    Nun beginnt der eigentliche Test mit allgemeinen Fragen. Der Experte fragt die
    Testperson zu ihrem Beruf und übergeht dann zu Fragen, wie oft die Testperson
    das Internet benutzt und wofür genau. Dazu werden auch Follow-Up Fragen gestellt. 
    \\
    Dann wird die Website getestet. Die Testperson muss die Seite genau anschauen und 
    sozusagen einmal ihre 'first impressions' wiedergeben. Was würde die Person auf dieser
    Seite machen? Was denkt sie wofür die ist?
    \\
    Im letzten Teil des Tests werden der Testperson spezifische Tasks gegeben. Der Experte
    erzählt ihr einen Use Case und gibt ihr eine Aufgabe, welche sie lösen muss und alles
    dokumentiert, was sie dabei macht und denkt. Insgesamt werden ihr zwei Tasks gegeben. 

    \item Usability Probleme
    
    1) Die Tesperson hätte sich einen Rechner direkt auf der Seite gewünscht, welche
    die Tarife anzeigt. Der User muss sich dort alles selbst zusammenrechnen
    wie viel 'ZipCar' ihm kosten würde.

    2) Der zweite Task 'find out about availability' stellte sich als eher schwierig heraus 
    für die Testperson. Als sie schlussendlich die Antwort fand, war sie nicht zufrieden mit 
    dem Resultat; es sei zu ungenau.

    3) Der Preis, den es anzeigt, nachdem man ein Auto gewählt hat, stimmt nicht mit
    dem Preis überein, den es anzeigt bei den Tarifen. Für die Testperson scheint das
    unehrlich und sie würde dieser Seite daher nicht weiter trauen. Es sollte somit 
    besser ersichtlich sein, dass der Preis variieren kann.
    
\end{itemize}