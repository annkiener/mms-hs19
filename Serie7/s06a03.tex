Beim \textit{Between Group Design} ist jeder 
Teilnehmer nur einer Versuchsbedingung ausgesetzt.
Die Anzahl Teilnehmer in der Gruppe entspricht somit der Anzahl Versuchsbedinungen.
Es könnte aber auch sein, dass die Interaktion zu schwierig ist für den Teilnehmenden und 
so oder so werden die Ergebnisse durch unterschiedliche Eigenschaften der Individuen verzerrt.
Jedoch hat man eine bessere Kontrolle über den Lerneffekt, da der Benutzer explizit nur in einer Bedingung arbeitet. 
Für den Teilnehmenden bedeutet dies aber auch eine viel kürzere Teilnahmezeit, weniger Müdigkeit
und Frustration als bei \textit{Within Group Design}. Dabei testet jeder Teilnehmer verschiedene
Versuchsbedinungen, dh. ein Teilnehmer erlebt mehrere Bedingungen. Man braucht also weniger Teilnehmer 
dafür. Und Verzerrungen durch den Lerneffekt im Within Group Design können
verringert werden, wenn die Reihenfolge, in der die Bedingungen ge-
testet werden, zwischen den Benutzern variiert wird. Auch spielt die Schwierigkeit der 
Interaktion nicht so eine Rolle weil man viele verschiedene Teilnehmer hat, was auch
Verzerrung des Ergebnisses verhindert.


