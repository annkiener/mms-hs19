Bei der textbasierten Kommunikation hat man in der Regel keine Back-Channels. So hat man
kaum Bestätigung, nur bei der nächsten Äusserung des Zuhörers. Ausserdem gibt es nicht immer
einen expliziten Zusammenhang zwischen verschiedenen Nachrichten, ausser man verweist auf eine frühere Nachricht.

Es fehlen bei der textbasierten Kommunikation auch verschiedene Kanäle (z.B. Mimik, Gestik, Körpersprache, Tonalität der Sprache, etc.).
Ohne diese Kanäle wird der Zustand des Sprechers und die Kraft der Botschaft nicht wirklich vermittelt.
Dadurch werden Ausfälle schwer oder kaum entdeckt und repariert.

