Für die nutzerbasierte Evaluation gibt es viele Möglichkeiten
wie Think Aloud, Umfragen, Interviews, etc. Man unterscheidet zunächst zwischen
Labor- und Feldstudien.
Für diese Aufgabe konzentrieren wir uns auf die Methode 'Think Aloud', welche
als Laborstudie durchgeführt wird.
\\ \\
Diese Studie wird in einem Labor vorgenommen wo eine Testperson eine Aufgabe gestellt
bekommt und gebeten wird ihr Vorgehen zur Lösung der Aufgabe zu kommentieren, also laut zu denken. 
Die Testperson erlaubt dabei einen Einblick in ihr Lösungsverhalten und erzählt alles möglichst ehrlich
und intuitiv. 
Ein Mitglied des Design Teams leitet die Aufgaben an unterstützt aber nicht.
Die Testperson bekommt diverse Tasks und wird währenddessen genau von dem Design Team beobachtet.
Sie muss selbst herausfinden, wie sie den gestellten Task meistert.
Der Test wird mit Video und Mouse-Tracking (Aufzeichnung des Klickverkaltens) 
aufgenommen und nachträglich ausgewertet.
\\ 
Jede der drei Navigationsarchitekturen wird von der Testperson durchgearbeitet, sodass
man einen direkten Vergleich bekommt, wie diese Person auf die verschiedenen Architekturen reagiert.
So kann man konkret die Stärken und Schwächen der jeweiligen Navigationsarchitekturen unterscheiden.