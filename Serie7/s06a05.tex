Für die nutzerbasierte Evaluation gibt es viele Möglichkeiten
wie Think Aloud, Umfragen, Interviews, etc. Für diese Aufgabe konzentrieren wir uns auf die Methode 'Think Aloud'.
\\
Diese Studie wird in einem Labor vorgenommen wo eine Testperson eine Aufgabe gestellt
bekommt und gebeten wird ihr Vorgehen zur Lösung der Aufgabe zu kommentieren, also laut zu denken. 
Die Testperson erlaubt dabei einen Einblick in ihr Lösungsverhalten, 
ein Mitglied des Design Teams leitet die Aufgaben an unterstützt aber nicht.

Jede der drei Navigationsarchitekturen wird von der Testperson durchgearbeitet, sodass
man einen direkten Vergleich bekommt, wie diese Person auf die verschiedenen Architekturen reagiert. 

Der Test wird mit Video und Mouse-Tracking (Aufzeichnung des Klickverkaltens) 
aufgenommen und nachträglich ausgewertet.