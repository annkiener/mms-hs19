 Bei einem \textit{Cognitive Walkthrough} liegt der Fokus darauf wie einfach es ist ein System durch Exploration zu erlenernen.
 Für den Walkthrough benötigt man einen Prototyp des Systems, Beschreibung der Aufgabe(n), die der Benutzer auf dem System ausführen
 soll, eine Liste der erforderlichen Aktionen und Beschreibung der Benutzer. Die Experten versuchen dann für jede Aktion die 
 folgenden vier Fragen zu beantworten:
 \\
Entspricht die Wirkung der Aktion dem Ziel des Benutzers?
\\
Sind die Aktionsmöglichkeiten sichtbar?
\\
Ist die Bedeutung der Aktionsmöglichkeiten klar?
\\
Wird der Benutzer das Feedback verstehen? 



Bei einem \textit{Post-Task Walkthrough} werden die Handlungen der Teilnehmer nach der Interaktion mit den Benutzern reflektiert.
Man fragt den Benutzer warum so gehandelt wurde.
Das Transkript wird dem Teilnehmer gezeigt und er wird dann vom Analysten dazu befragt.
Man möchte so Probleme mit einer Schnittstelle entdecken und beobachten, wie ein System tatsächlich genutzt wird. 
