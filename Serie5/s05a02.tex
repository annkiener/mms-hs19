\textit{Deduktives Denken} wird im Skript definiert als \textit{``Schluss vom
Allgemeinen auf das Besondere''}. Im Zusammenhang mit der Heuristik
\textit{Übereinstimmung zwischen System und der realen Welt} bedeudet dies, dass
ein Benutzer sich schneller in einer Applikaitons zurecht findet, falls diese
\textit{Phrasen und Konzepte} verwendet welche dem Benutzer bekannt sind. So ist
es dem Benutzer möglich schnell ein \textit{mentales Modell} zu entwickeln. 
\newline

\textit{Induktives Denken} wird im Skript definiert als \textit{``die
Verallgemeinerung (oder die Gewinnung von allgemeinen Aussagen) aus der
Betrachtung mehrerer Einzelfälle. ''}. Hier kann ein Zusammenhang mit
der \textbf{ISO-Norm} \textit{Erwartungskonform}, hergestellt werden. Eine
Applikation welche sich an \textit{Standards} hält ermöglicht es dem Benutzer
sich schnell zurecht zu finden, da er auf sein Wissen aus anderen Applikationen
zurückgreifen kann. Gleichzeitig kann der Benutzer auch mit dieser Applikation
neues erlernen welche in deren Applikationen wieder verwendet werden kann,
solang sie sich an die selben \textit{Standards} halten.

\textit{Abduktives Denken} wird im Skript definiert als \textit{``[...] die
Methode, mit der wir Erklärungen für die von uns beobachteten Ereignisse
ableiten. ''}. Vergleichen wir dies mit der Designrichtlinie \textit{Feedback}
von Normen, so sehen wir folgende Parallelen. Eine \textit{Handlung} führt oft zu einer
\textit{Veränderung}, halten wir uns an die Designrichlinie so geben wir dem Benutzer
eine Möglichkeit den \textit{neuen Status} leicht zu bestimmen und liefern so
eine \textit{Erklärung} was sich verändert hat. Desweiteren liefern wir bei
konrekter Ausführung der Designrichtlinie bereits im Voraus Informationen
bezüglich der \textit{Folgen von Handlungen} so das Zusammen mit dem Status ein
Benutzer sich sicher sein kann das der Zustand seiner Erwartung entspricht. 
 
