\begin{table}[h!]
    \centering
    \begin{tabular}{|l|l|}
        \hline
      \textbf{Grundsatz}  & Wiederherstellung \\
      \hline
      \textbf{Use Case}           & Einen falsch eingetragenen Termin wieder löschen  \\
      
      \textbf{Messmethode}     &     Anzahl wie oft das Interface den User irreführt         \\
      
      \textbf{Ist-Level}        &      Der User braucht vier Anläufe um den Termin zu löschen     \\
      
      \textbf{Schlechtester Fall}    &   Der User braucht drei Anläufe um den Termin zu löschen       \\
      
      \textbf{Geplantes Level}     &      Der User braucht zwei Anläufe um den Termin zu löschen        \\
      
      \textbf{Bester Fall}           &     Der User braucht einen Anlauf um den Termin zu löschen        \\
      \hline
    \end{tabular}

    \caption{Beispiel: Wiederherstellung}
    \label{tbl:wiederherstellung}

  \end{table}

\begin{table}[h!]
    \begin{tabular}{|l|l|}
        \hline
      \textbf{Grundsatz}  & Einsehbarkeit \\
      \hline
      \textbf{Use Case}           &    Alle Termine des Tages/der Woche/des Monats können vom User übersichtlich eingesehen werden \\
      
      \textbf{Messmethode}     &       Anzahl der Versuche die ein User braucht um seine gewünschte Übersicht zu erhalten       \\
      
      \textbf{Ist-Level}        &      Der User braucht vier Anläufe um seine gewünschte Einsicht zu haben       \\
      
      \textbf{Schlechtester Fall}    &       Der User braucht drei Anläufe um seine gewünschte Einsicht zu haben      \\
      
      \textbf{Geplantes Level}     &       Der User braucht zwei Anläufe um seine gewünschte Einsicht zu haben        \\
      
      \textbf{Bester Fall}           &      Der User braucht einen Anlauf um seine gewünschte Einsicht zu haben      \\
      \hline
    \end{tabular}

    \caption{Beispiel: Einsehbarkeit}
    \label{tbl:einsehbarkeit}

  \end{table}