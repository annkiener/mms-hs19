Die \textit{8 Goldenen Regeln von Ben Shneiderman} lauten wie folgt:
\begin{itemize}
  \item Konsistenz
  \item Informatives Feedback
  \item Abgeschlossenheit
  \item Fehler vermeiden
  \item Umkehrbarkeit
  \item Benutzerkontrolle gewährleisten
  \item Kurzzeitgedächtnis entlasten
  \item Universelle Benutzbarkeit
\end{itemize}

Dabei finden wir zum Teil grosse Übereinstimmung mit \textit{Nielsens Zehn
Heuristiken}. 

\begin{itemize}
  \item Umkehrbarkeit <-> Benutzerkontrolle und Freiheit
  \item Konsistenz <-> Konsistenz und Standards
  \item Fehler vermeiden <-> Fehlervermeidung 
  \item Kurzzeitgedächtnis entlasten <-> Erkennen statt Erinnern
  \item Universelle Benutzbarkeit <-> Flexibilität und Effizienz der Nutzung
  \item Informatives Feedback <-> Anwendern helfen, Fehler zu erkennen, zu diagnostizieren und zu beheben
\end{itemize}

Wie wir sehen werden die beiden Punkte \textit{Abgeschlossenheit} und
\textit{Benutzerkontrolle gewährleisten} nicht direkt auf die \textit{Zehn
Heuristiken} abgebildet. Gleichzeitig gibt es für \textit{Ästhetisches und
minimalistisches Design}, \textit{Übereinstimmung zwischen System und der
realen Welt}, \textit{Hilfe und Dokumentation} und \textit{Sichtbarkeit des
Systemstatus} kein direktes Pendant in den \textit{Goldenen Regeln}. 
