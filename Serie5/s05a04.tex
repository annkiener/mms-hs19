\begin{itemize}

    \item Struktur der Umfrage:

    Als erstes muss in einer Umfrage eine Einleitung stehen, welche angibt wieviel
    Zeit sie in anspruch nehmen wird, mögliche Tipps gibt zum Ausfüllen der Umfrage,
    sodass dem Benutzer klar ist, was er machen muss, wozu die Umfrage verwendet wird
    und ob sie anonym ist oder nicht. Ausserdem müssen Kontaktangaben gemacht werden,
    falls Probleme auftreten. 

    Um zu differenzieren zwischen USB Stick- und Cloud-Usern ist zu Beginn eine Frage bei dem nur eines
    der beiden Speichermethoden gewählt werden kann, sodass auch Kontingentfragen folgen können.
    Fragen zu den jeweiligen Speichermethoden sind jeweils die gleichen, aber es wird
    gruppiert in einen Teil für die Cloud und in einen Teil für den USB-Stick.

    Aus Motivationsgründen folgen allgemeine Fragen zum Benutzer, wie Alter, Geschlecht, 
    Ausbildung, Beruf, Computererfahrung, Hauptgerät, OS etc. erst am Schluss der Umfrage.

    \item Fragen:
    
    \begin{enumerate}[label=\arabic*)]
        \item Bevorzugen Sie USB Sticks oder speichern sie ihre Daten auf der Cloud?
        \item Gibt es spezifische Gründe wieso sie die Cloud/den USB-Stick bevorzugen? Wenn ja, welche?
        \item Gibt es Funktionalitäten die sie bei der Benutzung der Cloud/des USB-Sticks vermisst haben? Wenn ja, welche?
        \item Wie würden Sie die Qualität der Cloud/des USB-Sticks beschreiben?

            Schlecht    1   2   3   Sehr Hoch
        \item Wie sehr wurde die Cloud/der USB-Stick Ihren Bedürfnissen gerecht?

            überhaupt nicht gerecht     1   2   3   extrem gerecht

    \end{enumerate}

    \item Ja, wie bei der ersten Teilaufgaben bereits beschrieben, machen
    Kontigentfragen Sinn und sind sehr gut einsetzbar, gerade dann wenn man Präzisierungen von Antworten 
    haben möchte oder irgendwelche Begründungen für Präferenzen.


    \item Zur Erhöhung der Rücklaufquote kann man dem Benutzer das Rücksenden erleichtern (mit einem
    frankierten Umschlag), Erinnern oder Nachhaken über einen anderen Kommunikationskanal, oder eine
    Ersatzumfrage senden 2-4 Wochen nach dem ersten Versand.
    Es kann auch von Vorteil sein, wenn man zur Verteilung eine Adressliste hat, so wie es beispielsweise
    Umfragen gibt, die von Studierenden der Universität Bern ausgefüllt werden. Um dann
    die Eingeladenen noch mehr zu motivieren kann man einen Preis oder ein Gewinnsspiel erstellen.
    
\end{itemize}

 






 
