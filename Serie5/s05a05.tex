Tagebücher sind eine nützliche Methode zur Datenerfassung, 
insbesondere für Informationen, die sich im Laufe der Zeit ändern können.
Ein Benutzer kann seine Emotion und Stimmung auf eine unaufdringliche Weise wiedergeben
und ein Tagebuch ermöglicht eine sehr kleine Lücke zwischen dem Auftreten eines Ereignisses und der
Aufzeichnung des Ereignisses. Tagebücher sind ausserdem ortsungebunden und Vorfälle werden
besser aufzeichnen. 

Tagebuchstudien eignen sich vorallem für Technologien, die es noch nicht gibt, aber geben könnte -
so kann man unabhängig von der Technologie Kommunikations- oder
Informationsmuster untersuchen. Es macht auch Sinn eine Tagebuchstudie durchzuführen, für die Evaluation eines
Prototypen oder für schon existierende Technologien, welche aber verbessert werden sollen.